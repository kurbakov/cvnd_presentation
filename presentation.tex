\documentclass{beamer}

\usetheme{Boadilla}

\usepackage{listings}

\usepackage{hyperref}
\hypersetup{colorlinks=true}
\title{Udacity nanodegree overview}
\author{Kurbakov Dmytro}

\begin{document}
\begin{frame}
\titlepage
\end{frame}

\begin{frame}{Overview}
\tableofcontents
\end{frame}


\section{Nanodegree.. what does it mean?}

\begin{frame}
\begin{center}
\huge Nanodegree.. what does it mean?
\end{center}
\end{frame}

\begin{frame}[fragile]{Definition}
A Udacity Nanodegree® Program is a unique online educational offering designed to bridge the gap between learning and career goals. (Udacity)

Set of the learning courses with a focus on the practical side (Kurbakov D.). 

\end{frame}

\begin{frame}[fragile]{Who is a teacher?}
Short answer: Partners with Udacity Full time employees.

Partners;
\begin{itemize}
\item Amazon
\item Google
\item AT\&T
\item Lyft
\item Kaggle
\item NVidia
\item Facebook
\item IBM
\item ...
\end{itemize}

\end{frame}

\begin{frame}[fragile]{What nanodegreeas are available?}
Short answer: a lot.

The most interesting:
\begin{itemize}
\item Machine Learning Engineer
\item Deep learning
\item Deep reinforcement learning
\item Self-driving car Engineer
\item Artificial Intelligence
\item Flying car and autonomous flying engineer
\item Robotics software engineer
\item Computer vision
\item Cloud dev ops engineer
\item Cloud developer
\end{itemize}

\end{frame}


\begin{frame}{How the nanodegree structured?}

Core curriculum:
\begin{itemize}
\item Each of the curriculum has N parts (topics).
\item The part can be optional.
\item Non-optional part has the project, that is required to pass, if you want to graduate.
\item Non-optional part can have an optional project.
\item Only 1 deadline.
\end{itemize}

Extracurriculum:
\begin{itemize}
\item Each of the curriculum has N parts (topics).
\item All parts are optional
\item No projects
\item No deadlines
\end{itemize}

\end{frame}

\begin{frame}{Community}
Once you enroll into the course, you will have:
\begin{itemize}
\item Access to all courses and projects
\item Access to the slack
\item Slack channel with everyone who is doing the nenodegree with you
\item Assigned to you mentor (once a weekly group call, once a week 1:1 call)
\item Access to the career coach
\item All project reviwed by human
\end{itemize}

Once you graduate:
\begin{itemize}
\item Access to the alumni portal
\end{itemize}
\end{frame}

\begin{frame}{Price}
\begin{center}
Subscription modell

339 Euros per month
\end{center}
\end{frame}

\section{Computer vision nanodegree overview}

\begin{frame}
\begin{center}
\huge Computer vision nanodegree overview
\end{center}
\end{frame}

\begin{frame}{Overview}
\begin{enumerate}
\item Created in cooperation with NVidian and Affectiva.
\item Scheduled for 3 months with workload 10-15 hours per week.
\item 24 people on the slack group + mentor.
\end{enumerate}
\end{frame}

\begin{frame}{Mentor}
Ph.D. in Electrical and Electronical Engineering

Sr. Software Engineer in LG Electronics, located in San Jose, CA, USA.

Working on the simulator sor the testing autopilot (tracks).
\end{frame}


\begin{frame}{Course structure: Core curriculum}

Topics:
\begin{enumerate}
\item Introduction to the computer vision
\item Optional: cloud computing
\item Advanced computer vision and deep learning
\item object tracking and localization
\end{enumerate}
\end{frame}

\begin{frame}{Introduction to the computer vision}
\begin{enumerate}
\item Coding blue/green screen
\item Convolutional filters
\item ML in computer vision
\end{enumerate}
\end{frame}

\begin{frame}{Advanced compuer vision and deep learning}
\begin{enumerate}
\item YOLO
\item RNN
\item LSTM
\end{enumerate}
\end{frame}

\begin{frame}{Object tracking and lockalization}
\begin{itemize}
\item Undestanding of motion
\item Kalman filter
\item State of the motion and localization
\end{itemize}
\end{frame}


\begin{frame}{Projects}
Projects:
\begin{enumerate}
\item Facial keypoints detection
\item Image captioning
\item Landmark detection and tracking (SLAM)
\end{enumerate}
\end{frame}

\begin{frame}{Facial keypoints detection}
\end{frame}

\begin{frame}{Image captioning}
\end{frame}

\begin{frame}{Landmark detection and tracking (SLAM)}
\end{frame}


\section{Udacity vs. Coursera}

\begin{frame}{Udacity vs. Coursera}
\end{frame}

\section{Conclusions}
\begin{frame}
\begin{center}
\huge Conclusions
\end{center}
\end{frame}

\begin{frame}{Conclusions}
\end{frame}

\begin{frame}{Q\&A}
\end{frame}


\end{document}